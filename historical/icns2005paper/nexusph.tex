%----------------------------------------------------------------------
% Template article for ICNS2005 conference: 
% document class 'elsart4'
%
% This template should be used for all manuscripts submitted to the
% International Conference on Neutron Scattering 2005
% Darling Harbour. Sydney. Australia 27 November - 2 December, 2005
%
% Send any comments to the ICNS2005 Administration <nha@ansto.gov.au>
%----------------------------------------------------------------------
% The template contains a number of LaTeX commands of the form :
%
% \command{}
%
% In order to complete this abstract, fill in the blank fields between
% the curly braces with the requested information.
%
% e.g. in order to add an abstract title, replace 
% \title{}    with   \title{Evidence of Solitons in Tedium Diboride}
%
% The \author and \address commands can take an optional label 
% in square brackets of the form :
%
% \command[label]{}
%
% The text of the abstract should be inserted between the two commands
% \begin{abstract} and \end{abstract}.
%
% Please leave all commands in place even if you don't fill them in. 
% Do not alter commands which have non-blank arguments unless you are
% familiar with them.
%
% N.B. In LaTeX, all text following the % sign is treated as a comment.
%      Some commands, e.g. \thanks{}, have been "commented 
%      out".  Remove the % sign if you wish to use it.
%      Blank lines are ignored (except in the manuscript text where 
%      they will be treated as a paragraph break).
% 
%----------------------------------------------------------------------
% Invoke the LaTeX document class for these conference proceedings

\documentclass{elsart4}

%----------------------------------------------------------------------
% Install LaTeX packages to allow enhanced math formatting and 
% graphics installation.  These are the recommended packages.  If you 
% are experienced in using LaTeX, you may choose a different list.

\usepackage{graphicx}
\usepackage{amssymb}

%----------------------------------------------------------------------
% Start processing the document and front page

\begin{document}
\begin{frontmatter}

%----------------------------------------------------------------------
% Specify destination and version number of the manuscript


%----------------------------------------------------------------------
% Title of manuscript

\title{The State of the NeXus Data Format}

%----------------------------------------------------------------------
% List of authors
%
% List each author using a separate \author{} command
%
% If there is more than one author address, add a label to each author
% of the form \author[label]{name}.  This label should be identical to
% the corresponding label provided with the \address command.
%
% e.g. if there are three authors from two institutions in USA and 
% France, you can link them to their respective addresses, using
%
% \author[US]{John Doe}
% \author[US,FR]{Jane Doe}
% \author[FR]{Jean Dupont}
% \address[US]{University of Life, Somewhere, USA}
% \address[FR]{Universite de la Vie, Quelque Part, France}
%
% N.B. Unlike the document class used for abstract submissions, it is
% possible to have the author associated with more than one address,
% as shown in the example above.
%

\author{NeXus International Advisory Committee\corauthref{1}}

%----------------------------------------------------------------------
% List of addresses
%
% If there is more than one address, list each using a separate 
% \address command using a label to link it to the respective author
% as described above
 
\address{}

%----------------------------------------------------------------------
% Title page footnotes
%
% If you need to add qualifying information to any of the authors, 
% use the \thanksref{} command within the \author command.  The 
% argument is the label of a corresponding \thanks[label]{text}
% command which contains the footnote text
%
% e.g. you can acknowledge a funding authority for John Doe, using
%
% \author{John Doe\thanksref{ABC}}
% \thanks[ABC]{This work was supported by Institute of Unphysical 
%    Phenomena under contract no. ABC-123}
%

%\thanks[]{}

%----------------------------------------------------------------------
% Contact Information
%
% Add the complete postal address, telephone number, fax number, and
% email address of the corresponding author as a special footnote using
% the \corauth[]{} command.  This works in a similar way to the \thanks 
% command.  Add the \corauthref{} command within the \author command.
% The argument is the label of a corresponding \corauth[label]{text}
% command which contains the contact information.  Prefix the text with
% Corresponding Author:
%
% e.g. if the contact author is John Doe,
%
% \author{John Doe\corauthref{1}}
% \corauth[1]{Corresponding Author: University of Life, 123 Some St.,
%    Somewhere, MI 12345, USA.  Phone: (555) 555-5555 
%    Fax: (555) 555-7777, Email: JDoe@uol.edu}
%

\corauth[1]{Corresponding Author: Mark K\"onnecke, 
  Laboratory for Development and Methods,
Paul Scherrer Institute, 5232 Villigen-PSI, Switzerland, 
Phone: +41-56-3102512, Fax: +41-56-3102939,Email: Mark.Koennecke@psi.ch}

%----------------------------------------------------------------------
% Text of abstract

\begin{abstract}

NeXus is an effort by an international group of scientists to define 
 a common data exchange format for neutron, muon and x-ray scattering.   
 NeXus has six levels: a physical file format, a file structure, rules 
 for storing individual data items in a file, a dictionary of names,  
 instrument definitions and an Application Programming Interface (API) to NeXus 
 files. The authors will present the large steps forward which have been 
 made both with instrument definitions and the NeXus-API. 


Authors to 
 NeXus are:\begin{it}
Freddie Akeroyd, ISIS Facility, Rutherford Appleton Laboratory, UK;
Stuart Cambell, Diamond Light Source, UK;
Stephen Cottrell, ISIS Facility, Rutherford Appleton Laboratory, UK;
Matthias Drochner, Forschungszentrum J\"ulich, Germany;
Nick Hauser, Australian Nuclear Science and Technology Organisation, Australia;
Ron Ghosh, Institute Laue Langevin, Grenoble, France;
Andrew G\"otz, European Synchrotron Radiation Facility, France;
Hartmut Gilde, FRM2, Technische Universit\"at M\"unchen, Germany;
Przemek Klosowski, National Institute of Standards and Technology, USA; 
Mark K\"onnecke, Paul Scherrer Institute, Switzerland;
Ray Osborn, Intense Pulsed Neutron Source, ANL, USA;
Toshiya Otomo, KEK, Japan;
Peter Peterson, Spallation Neutron Source, USA;
Thomas Proffen, Lujan Neutron Scattering Center, USA.
\end{it}
\end{abstract}

%----------------------------------------------------------------------
% Manuscript keywords
%
% Please give two or three keywords in the form: keyword \sep keyword
% e.g. NMR \sep superconductivity
%
% NB The syntax is different from the abstract document class

\begin{keyword}
data format \sep data analysis \sep data management

\end{keyword}

%----------------------------------------------------------------------
% End of front page

\end{frontmatter}

%----------------------------------------------------------------------
% Manuscript text
%
% Fill in the following space with the manuscript text.
%
% A number of LaTeX commands may be invoked in this space, e.g.
%
% \section{} : to insert a new section title
% \label{}   : to label the numbered section for use in \ref{}
% \cite{}    : to add a reference using the label in \bibitem{}
% 
% A number of LaTeX environments may be used, e.g. 
% \begin{equation}
%     An equation inserted here will be automatically numbered
% \end{equation}  
%
% Please refer to other LaTeX documentation for help on using these
% environments.
\section{Introduction}
As of now, most major facilities choose to store the data measured at 
 neutron, x-ray or muon instruments in a number of different home
 grown data formats. This situation makes the life of the travelling
 scientist more difficult than it needs to be because they have to
 cater for many different data formats while performing data reduction
 and analysis. Moreover, the existence of many different data formats
 is a hurdle to sharing data reduction and analysis software. To
 improve this situation a single, common data exchange format is 
 proposed for both raw and preprocessed data. This
 proposal is NeXus, the NEutron, X-ray, $\mu$ (muon) Science data format.    

\section{NeXus Guiding Principles}
The NeXus team tried to follow a few guiding principles while
designing the format:
\begin{itemize}
\item Portability across common computing platforms. 
\item Self describing. It must be possible to deduce the content of
  the file from data in the file alone.
\item Extensible. It must be possible to add data to the file without
  breaking code based on the standard. 
\item Flexibility in data organisation. This is required to cope with the
  plethora of different instruments addressed.
\item Upcoming high powered neutron and x-ray sources demand efficient
  data storage.  
\item Completeness. All data necessary for common data treatment tasks 
  should be stored in one file. 
\item Easy Access. The user should be protected from writing low level
  parsers.
\item Facilitate Automatic Plotting. 
\item Availability in the public domain. 
\end{itemize}

\section{NeXus Overview}
In order to meet these guidelines the NeXus team developed a proposal
consisting of six levels:
\begin{itemize}
\item An underlying Physical File Format.
\item A file organsisation.
\item A programming interface (API) to access data files.
\item Rules for storing data items in a file.
\item A collection of instrument component definitions.
\item A collection of instrument definitions.
\end{itemize}

\section{Physical File Format and the NeXus-API}
With NeXus--API version 3.0 (released: July 2005) three different
physical file formats are supported:
\begin{itemize}
\item HDF--4\cite{hdf4}
\item HDF--5\cite{hdf5}
\item XML
\end{itemize}
Rather than inventing yet another binary file format the NeXus team
chose an existing scientific data format, HDF, as one of its
underlying physical file formats. When a newer version of HDF, HDF-5,
appeared support for this format was added, too. These two binary file
formats allow for the efficient storage of large data sets and support
transparent on the fly compression and decompression of data while
reading or writing. The HDF file formats are also natively supported by many
commercial and freeware data treatment packages--such tools can
instantly be used to treat NeXus files. XML is a structured ASCII file
format. Support for XML was added to cater for those
scientists who wish to be able to edit their data manually. All three
file formats allow for structured data storage in the file. Not only
scientific datasets (multi-dimensional arrays of numbers) are
supported, but also grouping constructs which allow for ordering
elements in a hierarchical manner, much like in a file system.


Access to all three physical file formats is provided through the
NeXus--API. With the roughly 30 functions of this API a user can
construct and navigate a NeXus file's hierarchy, write and read data and
inquire meta information without having to even know about the
underlying physical file format. The NeXus--API is written in portable
ANSI-C. Language bindings exist for FORTRAN-77, FORTRAN-90, Java,
python, Tcl and through a SWIG\cite{swig} interface to a plethora of
common scripting languages. The NeXus--API comes with a small set of
utilities including a file browser and nxconvert, a utility which
allows conversion between all three physical file formats.  
     
\section{NeXus File Structure}
For an overview of the structure of a NeXus file, see table 1.
\begin{table}
\label{struc}
\begin{tabular}{|p{1cm}|p{6.5cm}|}
\hline
\multicolumn{2}{|l|}{NXroot}\\
\hline
The root level of a NeXus file & 
\begin{tabular}{p{1.5cm}|p{5.0cm}}
\multicolumn{2}{l}{NXentry}\\
\hline
All data belonging to one scan or run. A given NeXus file
can contain multiple related scans or runs &
\begin{tabular}{p{1.5cm}|p{3.3cm}}
\multicolumn{2}{l}{NXinstrument}\\
\hline
The data needed to describe an instrument. Contains groups for
each relevant instrument component&
\begin{tabular}{p{3.3cm}}
NXsource\\
\hline
NXmoderator\\
\hline
NXvelocity\_selector\\
\hline
NXcollimator\\
\hline
NXattenuator\\
\hline
NXdetector\\
\end{tabular}\\
\hline
\multicolumn{2}{l}{NXsample}\\
\multicolumn{2}{p{3.3cm}}{All the information about the sample}\\
\hline
\multicolumn{2}{l}{NXmonitor}\\
\multicolumn{2}{l}{Control monitor}\\
\hline
\multicolumn{2}{l}{NXuser}\\
\multicolumn{2}{l}{User information}\\
\hline
\multicolumn{2}{l}{NXdata}\\
\multicolumn{2}{p{3.3cm}}{Links to plottable data in the NXdetector group --
 one instance for each detector bank. This provides support for generating a
typical plot automatically}\\
\end{tabular}\\
\end{tabular}\\
\hline
\end{tabular}
\caption{Overview of the structure of a NeXus file}
\end{table}
Each NeXus group (or directory) has both a type (class) and a
name; NeXus only standardizes the class names. At the root level of
each NeXus file there is some global information and one to $n$ NXentry
groups. Each NXentry group holds all the data related to one scan or
run. NXentry is the NeXus construct for holding multiple related data sets in one
file. Within each NXentry there are further groups: NXinstrument,
NXsample, NXuser, NXmonitor and NXdata and possibly additional
groups. NXinstrument itself contains further groups which represent the
building blocks of the instrument--each of these groups will contain data
describing the component and its position within the instrument:
\begin{itemize} 
\item the NXdetector group contains detector information and also the counts 
\item the NXsample group holds the sample information
\item the NXmonitor group holds monitor information
\item the NXdata group contains plottable data i.e. a link to the counts along 
   with axis information which an automatic tool can use
to display a default plot of the data. 
\end{itemize} 

The above structure would seem to require
the duplication of possibly large data sets. This is not the case as
both the file format and the NeXus--API support the concept of links,
i.e. references to data elements already written. This is very similar to
links in a UNIX file system. 

Additional groups may be present in an NXentry group which describe
event based data, logged data items, processing information or the 
intent of the data.

Within groups, datasets are stored which describe the
corresponding component - these datasets can also have associated attributes. 
A standard attribute which has to be written is the units used. 
For unit names we
use the conventions established by the Udunits\cite{udu} unit
conversion program. The axes datasets describing the dimensions of a 
 multi dimensional dataset are stored in the same group as the
 dataset. A convention allows one to associate the dimensions of a multi 
 dimensional dataset with the appropriate datasets describing the axis. 

For defining the position of a component within an instrument two
schemes exist: a simple one where positions are defined by distances
to the previous component, polar\_angle (in most cases equivalent to
two theta) and azimuthal angle. For the distances, the sample is at zero,
towards the source is negative, towards the detector is
positive. There also exists a more sophisticated system using
NXgeometry and its sub classes. The system is based on the coordinate
system of McStas\cite{mcstas} and has been included in order to meet 
the demands of the instrument simulation community.  

\section{NeXus Component and Instrument Definitions}
The NeXus International Advisory Committee (NIAC) was founded to oversee
the development of NeXus and further the creation of both component
definitions and instrument definitions. The following component
definitions now exist: 
\begin{itemize}
\item NXentry
\begin{itemize}
\item NXinstrument
\begin{itemize}
\item NXsource
\item NXmoderator
\item NXcrystal
\item NXchopper
\item NXguide
\item NXcollimator
\item NXaperture
\item NXfilter
\item NXattenuator
\item NXflipper
\item NXmirror
\item NXdetector
\item NXbeam\_stop
\end{itemize}
\item NXsample
\item NXmonitor
\item NXdata
\item NXevent\_data
\item NXuser
\item NXprocess
\item NXcharacterizations
\end{itemize}
\end{itemize}
In addition there are groups which may appear at any appropriate level
in the hierarchy:
\begin{description}
\item[NXlog] for logging variables, for example temperature 
\item[NXnote] for free text notes 
\item[NXbeam] for summarizing the status of the incoming or outgoing 
 neutron beam. 
\item[NXgeometry] with subgroups NXtranslation,
NXshape, NXorientation for defining instrument component positions
very accuratetly. 
\item[NXenvironment] with subgroup NXsensor for handling sample 
 environment controllers.
\end{description} 


So far the following instrument definitions have been approved:
\begin{itemize}
\item Time of flight neutron powder diffractometer
\item Monochromatic triple axis spectrometer
\item Monochromatic small angle scattering instrument
\item Time-of-flight Neutron Reflectometer.
\item Direct geometry time-of-flight spectrometer.
\end{itemize}
More are to follow.

\section{Conclusion}
At the last count, 14 major facilities world wide have commited
themselves to use NeXus for data storage and more than 30 NeXus aware
data treatment programs are already available. This demonstrates that the
NeXus file format is gaining widespread acceptance. The NeXus--API has
proven mature enough to write and process more then 500 000 files, 
especially at PSI. For more information,
do not hesitate to consult the NeXus WWW--site\cite{nx} or to contact
the members of the NIAC.  

%----------------------------------------------------------------------
% Reference section
%
% List each reference with a separate \bibitem{} command.  The
% argument contains the label that is used in the \cite{} command
% in the main text
%
% e.g.
%
%    This follows our pioneering work on TdB2\cite{TdB2}.
%
% \bibitem{TdB2}
% J. Doe, J. Doe, and J. Dupont, J. Irrep. Res. 10 (2000) 1000.

\begin{thebibliography}{00}

\bibitem{hdf4} http://hdf.ncsa.uiuc.edu/hdf4.html
\bibitem{hdf5} http://hdf.ncsa.uiuc.edu/HDF5
\bibitem{swig} http://www.swig.org
\bibitem{udu} http://www.unidata.ucar.edu/packages/udunits
\bibitem{nx} http://www.nexus.anl.gov; http://www.nexus.anl.gov/mediawiki
\bibitem{mcstas}K. Lefmann, K. Nielsen, Neutron News. 10 (1999) 20

\end{thebibliography}

%----------------------------------------------------------------------
% Figures and Tables
%
% Insert figures and tables at the end of the document unless you
% are familiar with the LaTeX positional options.
%
% \begin{figure}
%     \centering
%     \includegraphics{filename.eps}
%     \caption{Insert figure caption here} 
% \end{figure}  
%
% \begin{table}
%     \centering
%     \begin{tabular}
%     Insert table here
%     \end{tabular}
%     \caption{Insert table caption here}
% \end{table}  
%
% Please refer to other LaTeX documentation for help on using these
% environments.

%----------------------------------------------------------------------
% Terminate document

\end{document}

%----------------------------------------------------------------------

